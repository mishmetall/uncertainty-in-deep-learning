\documentclass[a4paper,14pt,href]{article}

% Используем нестандартный размер шрифта
\usepackage{extsizes}

% Делаем отступ для первого параграфа
\usepackage{indentfirst}

% Для поддержки поиска по pdf документу
\usepackage{cmap}

% Поддержка кириллицы
\usepackage[T2A]{fontenc}
\usepackage[utf8]{inputenc}
\usepackage[english,russian]{babel}

% Поддержка списков
\usepackage{enumerate}

% Графический пакет
\usepackage[final]{graphicx}
\usepackage{epstopdf}
\usepackage{tikz}

% Математические шрифты AMS
\usepackage{amstext, amssymb, amsmath}

% Поддержка гиперссылок
\usepackage{url}

% Задаем полуторный межстрочный интервал
\linespread{1.3}

% Задаем глубину оглавления
\setcounter{tocdepth}{2}

% Путь к изображениям, по умолчанию
\graphicspath{{images/}}

% Задаем отступ абзаца
\setlength{\parindent}{1.25cm}



\title{Неопределенность в машинном обученииределенность в машинном обучении}

\author{Ярин Гал}


Департамент инженерии Кэмбриджского университета


\part{Реферат}

\section{Глубокое обучение привлекло огромное внимание исследователей из множества областей информационной инженерии, таких как ИИ, машинное зрение, обработка языка [Kalchbrenner and Blunsom, 2013; Krizhevsky et al., 2012; Mnih et al., 2013], а также из более традиционных областей знаний, таких как физика, биология и промышленность [Anjos et al., 2015; Baldi et al., 2014; Bergmann et al., 2014]. Нейронные сети, средства обработки изображений, такие как сверточные нейронные сети, модели для обработки последовательностей, такие как рекуррентные нейронные сети, средства для регуляризации, такие как дропаут, нашли широкое применение. Однако, в таких области как физика, биология, промышленность выражение неопределенности модели является критически важным [Ghahramani, 2015; Krzywinski and Altman, 2013]. Начиная с недавнего сдвига этих областей в сторону использования Баесовской неопределенности [Herzog and Ostwald, 2013; Nuzzo, 2014; Trafimow and Marks, 2015], такая потребность возникла и у глубокого обучения.}
В данной работе мы разработали метод для практического определения неопределенности в глубоком обучении, преподнося методы глубокого обучения как Баесовские модели без изменения как самих моделей таки и методов оптимизации. В первой части этой работы мы разработаем теорию метода, а также предоставим иллюстрации и примеры использования. Мы свяжем приближенный вывод в Баесовских моделях с дропаутом и другими стохастическими методами регуляризации, а также оценим данное приближение эмпирически. Мы приведем примеры приложений, которые проистекают из связи между современным глубоким обучением и баесовскими моделями такими как активное обучение на изображениях, а также эффективное с точки зрения данных обучение с подкреплением. Далее мы демонстрируем метод на практике путем рассмотрения ряда приложений, использующих на практике разработанный метод, таких как обработка речи, медицинская диагностика, биоинформатика, обработка изображений, беспилотное управление автомобилем. Во второй части данной работы мы исследуем данные проистекающие от связи Баесовского моделирования с глубоким обучением и их теоретические следствия. Мы исследуем что же определяет свойство неопределенности, анализируем приближенный вывод аналитически в линейных случаях и теоретически исследуем различные априорные вероятности, такие как вероятности всплесков и плато.